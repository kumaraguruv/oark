\documentclass[12pt,a4paper,english]{book}
\usepackage{url}
\usepackage{color}
\usepackage{graphicx}
\usepackage{hyperref}
\usepackage{fancyhdr}
\usepackage{listings}
\usepackage[autocite=inline, labelalpha=true, style=alphabetic]{biblatex}
\usepackage{makeidx}
\usepackage{tocbibind}
\usepackage[lmargin=2.5cm, rmargin=2.5cm,tmargin=2.5cm,bmargin=2.5cm]{geometry}
\usepackage[compact]{titlesec}

\nocite{*}

\titleformat{\chapter}[display]
  {\bfseries\Huge}
  {\filright\Large\chaptertitlename\ \thechapter}
  {0mm}{\filright}
\titlespacing*{\chapter}
  {0pt}{-10pt}{15pt}

\makeindex

\bibliography{mybib}


\newcommand{\keyword}[1]{\index{#1}#1}
\newcommand{\ocite}[1]{\footfullcite{#1}}
\newcommand{\oscite}[1]{\cite{#1}}
\newcommand{\paraph}{\paragraph{}}

\let\cleardoublepage\clearpage

\pagestyle{fancy}
\fancyhead{}
\fancyfoot{}
\fancyfoot[C]{\thepage}
\fancyhead[LE,RO]{\slshape \rightmark}
\fancyhead[LO,RE]{\slshape \leftmark}

\renewcommand{\headrulewidth}{0pt}
\renewcommand{\footrulewidth}{0pt}

\definecolor{darkgray}{rgb}{0.47,0.79,0.47}

\hypersetup{
  colorlinks = true,
  linkcolor=blue,   % color of internal links
  citecolor=blue,   % color of links to bibliography
  urlcolor=blue,    % color of external links
  pagebackref=true,
  implicit=false,
  bookmarks=true,
  bookmarksopen=true,
  pdfdisplaydoctitle=true
}



\begin{document}

\lstset{ %
language=Octave,                % choose the language of the code
basicstyle=\footnotesize,      % the size of the fonts that are used for the code
numbers=left,                   % where to put the line-numbers
numberstyle=\footnotesize,      % the size of the fonts that are used for the line-numbers
stepnumber=0,                   % the step between two line-numbers. If it's 1 each line
                                % will be numbered
numbersep=5pt,                  % how far the line-numbers are from the code
backgroundcolor=\color{white},  % choose the background color. You must add \usepackage{color}
showspaces=false,               % show spaces adding particular underscores
showstringspaces=false,         % underline spaces within strings
showtabs=false,                 % show tabs within strings adding particular underscores
frame=tb,
tabsize=2,	                % sets default tabsize to 2 spaces
captionpos=b,                   % sets the caption-position to bottom
breaklines=true,                % sets automatic line breaking
breakatwhitespace=false,        % sets if automatic breaks should only happen at whitespace
title=\lstname,                 % show the filename of files included with \lstinputlisting;
                                % also try caption instead of title
escapeinside={\%*}{*)},         % if you want to add a comment within your code
morekeywords={*,...},            % if you want to add more keywords to the set
keywordstyle={\color{blue}},
commentstyle={\color{darkgray}},
stringstyle={\color{red}},
framexleftmargin=3pt,
framexrightmargin=3pt,
framextopmargin=3pt,
framexbottommargin=3pt
}

\title{The oark book: \url{http://code.google.com/p/oark/}\newline\begin{figure} [h]
\begin {center}
\includegraphics[width=0.2\textwidth]{oark.jpg}
\end {center}
\end{figure} \newline The Open Source Anti Rootkit}
\author{David Reguera Garcia aka Dreg - Dreg@fr33project.org \\\\Co-authors: -}
\maketitle
\date{}

\titlespacing*{\chapter}{0pt}{-40pt}{15pt}
\fancyhead[LE,RO]{}
\tableofcontents
\fancyhead[LE,RO]{\slshape \rightmark}

\chapter{Call Gates and GDT/LDT}

\section{GDT/LDT}
\paraph{}
The Global Descriptor Table or GDT (extracted from Wikipedia\ocite{wikglb2010}) is a data structure used by Intel x86-family processors in order to define the characteristics of the various memory areas used during program execution, for example the base address, the size and access privileges like executability and writability. These memory areas are called segments in Intel terminology. \ocite{Int6432sofdvman3A}

\paraph{}
The GDT can hold things other than segment descriptors as well. Every 8-byte entry in the GDT is a descriptor, but these can be Task State Segment (or TSS) descriptors, Local Descriptor Table (LDT) descriptors, or Call Gate descriptors. The last one, Call Gates, are particularly important for transferring control between x86 privilege levels although this mechanism is not used on most modern operating systems.

\paraph{}
There is also an LDT or Local Descriptor Table. The LDT is supposed to contain memory segments which are private to a specific program, while the GDT is supposed to contain global segments. The x86 processors contain facilities for automatically switching the current LDT on specific machine events, but no facilities for automatically switching the GDT.

\paraph{}
Every memory access which a program can perform always goes through a segment. On the 386 processor and later, because of 32-bit segment offsets and limits, it is possible to make segments cover the entire addressable memory, which makes segment-relative addressing transparent to the user.

\paraph{}
In order to reference a segment, a program must use its index inside the GDT or the LDT. Such an index is called a segment selector or selector in short. The selector must generally be loaded into a segment register to be used. Apart from the machine instructions which allow one to set/get the position of the GDT (and of the Interrupt Descriptor Table) in memory, every machine instruction referencing memory has an implicit Segment Register, occasionally two. Most of the time this Segment Register can be overridden by adding a Segment Prefix before the instruction.

\paraph{}
Loading a selector into a segment register automatically reads the GDT or the LDT and stores the properties of the segment inside the processor itself. Subsequent modifications to the GDT or LDT will not be effective unless the segment register is reloaded.

\section{Call Gates}

\paraph{}
A Call Gate \oscite{Int6432sofdvman3A} is a mechanism in the Intel x86 architecture to change privilege levels of the CPU when running a predefined function that is called by the instruction CALL/JMP FAR.

\paraph{}
A call to a Call Gate allows you to obtain higher privileges than the current, for example we can execute a routine in ring0 using a CALL FAR in ring3. A Call Gate is an entry in the GDT (Global Descriptor Table) or LDT (Local Descriptor Table). There are a GDT for each CORE, and each GDT can have one or more LDTs \ocite{dregrkar2010}.

\paraph{}
Windows doesn't use Call Gate for anything special, but there are malware, as the worm Gurong.A \ocite{w32guronga2006} , that installs a Call Gate via DevicePhysicalMemory to execute code on ring0. An article that talks about it is "Playing with Windows/dev/(k)mem" \ocite{crazylplayph2002}.

\paraph{}
Nowadays we can't easily access to /Device/PhysicalMemory, I recommend reading the presentation by Alex Ionescu at RECON 2006 "Subverting Windows 2003 SP1 Kernel Integrity Protection" \ocite{alexionsubwin2006}. Also, there are examples \ocite{saccocallg2006} in the wired that use the API ZwSystemDebugControl \ocite{undocntinternals} to install a Call Gate, but Ionescu's article says that it doesn't work nowadays (although there are techniques to reactivate them).

\paraph{}
Gynvael and j00ru made a Call-Gate mechanism in kernel/driver exploit development on Windows \ocite{j00rugyngdtldt2010}, or, to be more precise, to use a write-what-where condition to convert a custom LDT entry into a Call-Gate (this can be done by modifying just one byte), and using the Call-Gate to elevate the code privilege from user-land to ring0.

\paraph{}
David Reguera Garcia aka Dreg made a Call Gate detector for the free anti-rootkit Rootkit Unhooker \ocite{diableprkusr12010}. The next releases have new features to detect other new stuff like the 'new LDT Forward to user mode attack' (published also in the Gynvael and j00ru's paper). With this attack, an attacker can add a Call Gate or other descriptor to LDT without restrictions from user mode.

\paraph{}
An entry in the GDT/LDT looks like this:
\lstset{language=C,caption=GDT/LDT Descriptor structure}
\begin{lstlisting}
typedef struct _SEG_DESCRIPTOR
{
    WORD size_00_15;
    WORD baseAddress_00_15;
    WORD baseAddress_16_23:8;
    WORD type:4;
    WORD sFlag:1;
    WORD dpl:2;
    WORD pFlag:1;
    WORD size_16_19:4;
    WORD notUsed:1;
    WORD lFlag:1;
    WORD DB:1;
    WORD gFlag:1;
    WORD baseAddress_24_31:8;
} SEG_DESCRIPTOR, *PSEG_DESCRIPTOR;
\end{lstlisting}

\paraph{}
A Call Gate is an entry type in the GDT/LDT which has the following appearance:
\lstset{language=C,caption=Call Gate Descriptor structure}
\begin{lstlisting}
typedef struct _CALL_GATE_DESCRIPTOR
{
    WORD offset_00_15;
    WORD selector;
    WORD argCount:5;
    WORD zeroes:3;
    WORD type:4;
    WORD sFlag:1;
    WORD dpl:2;
    WORD pFlag:1;
    WORD offset_16_31;
} CALL_GATE_DESCRIPTOR, *PCALL_GATE_DESCRIPTOR;
\end{lstlisting}

\begin{itemize}
\item {{\bf offset\_00\_15:} is the bottom of the address of the routine to be executed in ring0, {\bf offset\_16\_31} is the top.}
\item {{\bf selector:} specifies the code segment with the value KGDT\_R0\_CODE (0x8), the routine will run ring0 privileges.}
\item {{\bf argCount:} the number of arguments of the routine in DWORDs.}
\item {{\bf type:} the descriptor type for a 32-bit Call Gate needs the value 0xC}
\item {{\bf dpl:} minimum privileges that the code must have to call the routine, in this case 0x3, because it will be called by the routine ring3}
\end{itemize}

\paraph{}
To create a Call Gate we can follow the following steps:
\begin{enumerate}
\item {Build the Call Gate that points to our routine.}
\item {Set the code only in a core (remember: there are a GDT for each CORE.}
\item {Read the GDTR register in order to find the GDT address and the size using SGDT instruction:

\lstset{language=C,caption=GDTR register}
\begin{lstlisting}
typedef struct _GDTR
{
    WORD nBytes;
    DWORD baseAddress;
} GDTR;
\end{lstlisting}

We can obtain the number of entries (number of GDT descriptors) with GDTR.nBytes/8.
}
\item {Find a free entry in the GDT/LDT.}
\item {Write the Call Gate descriptor.}
\item { To call the Call Gate is only necessary to make a CALL/JMP FAR to the GDT/LDT selector:
\begin{itemize}
\item { ie if we've introduced the Call Gate at the entry {\bf 100} of the {\bf GDT}, the user space application must execute a CALL/JMP FAR {\bf 0x320}:00000000. 0x320 is in binary 1100100 {\bf 0} 00, then the entry is:1100100 (100 in binary is 1100100), {\bf TI=0} (entry is in {\bf GDT}) RPL=00.. The other part of the FAR CALL is not useful but must be in the instruction.}
\item { ie if we've introduced the Call Gate at the entry {\bf 100} of the {\bf LDT}, the user space application must execute a CALL/JMP FAR {\bf 0x324}:00000000. 0x324 is in binary 1100100 {\bf 1} 00, then the entry is:1100100 (100 in binary is 1100100), {\bf TI=1} (entry is in {\bf LDT}) RPL=00..}
\end{itemize}
}

\end{enumerate}

\subsection{An example scenario}
\paraph{}
a

\chapter{PEB Hooking}
\paraph{}
This chapter's content.. \keyword{keyword1}

\fancyhead[LE,RO]{}
\printbibliography[heading=bibintoc]

\clearpage
\phantomsection
\printindex


\end{document}

























